% Options for packages loaded elsewhere
% Options for packages loaded elsewhere
\PassOptionsToPackage{unicode}{hyperref}
\PassOptionsToPackage{hyphens}{url}
\PassOptionsToPackage{dvipsnames,svgnames,x11names}{xcolor}
%
\documentclass[
  letterpaper,
  DIV=11,
  numbers=noendperiod]{scrartcl}
\usepackage{xcolor}
\usepackage{amsmath,amssymb}
\setcounter{secnumdepth}{5}
\usepackage{iftex}
\ifPDFTeX
  \usepackage[T1]{fontenc}
  \usepackage[utf8]{inputenc}
  \usepackage{textcomp} % provide euro and other symbols
\else % if luatex or xetex
  \usepackage{unicode-math} % this also loads fontspec
  \defaultfontfeatures{Scale=MatchLowercase}
  \defaultfontfeatures[\rmfamily]{Ligatures=TeX,Scale=1}
\fi
\usepackage{lmodern}
\ifPDFTeX\else
  % xetex/luatex font selection
\fi
% Use upquote if available, for straight quotes in verbatim environments
\IfFileExists{upquote.sty}{\usepackage{upquote}}{}
\IfFileExists{microtype.sty}{% use microtype if available
  \usepackage[]{microtype}
  \UseMicrotypeSet[protrusion]{basicmath} % disable protrusion for tt fonts
}{}
\makeatletter
\@ifundefined{KOMAClassName}{% if non-KOMA class
  \IfFileExists{parskip.sty}{%
    \usepackage{parskip}
  }{% else
    \setlength{\parindent}{0pt}
    \setlength{\parskip}{6pt plus 2pt minus 1pt}}
}{% if KOMA class
  \KOMAoptions{parskip=half}}
\makeatother
% Make \paragraph and \subparagraph free-standing
\makeatletter
\ifx\paragraph\undefined\else
  \let\oldparagraph\paragraph
  \renewcommand{\paragraph}{
    \@ifstar
      \xxxParagraphStar
      \xxxParagraphNoStar
  }
  \newcommand{\xxxParagraphStar}[1]{\oldparagraph*{#1}\mbox{}}
  \newcommand{\xxxParagraphNoStar}[1]{\oldparagraph{#1}\mbox{}}
\fi
\ifx\subparagraph\undefined\else
  \let\oldsubparagraph\subparagraph
  \renewcommand{\subparagraph}{
    \@ifstar
      \xxxSubParagraphStar
      \xxxSubParagraphNoStar
  }
  \newcommand{\xxxSubParagraphStar}[1]{\oldsubparagraph*{#1}\mbox{}}
  \newcommand{\xxxSubParagraphNoStar}[1]{\oldsubparagraph{#1}\mbox{}}
\fi
\makeatother

\usepackage{color}
\usepackage{fancyvrb}
\newcommand{\VerbBar}{|}
\newcommand{\VERB}{\Verb[commandchars=\\\{\}]}
\DefineVerbatimEnvironment{Highlighting}{Verbatim}{commandchars=\\\{\}}
% Add ',fontsize=\small' for more characters per line
\usepackage{framed}
\definecolor{shadecolor}{RGB}{241,243,245}
\newenvironment{Shaded}{\begin{snugshade}}{\end{snugshade}}
\newcommand{\AlertTok}[1]{\textcolor[rgb]{0.68,0.00,0.00}{#1}}
\newcommand{\AnnotationTok}[1]{\textcolor[rgb]{0.37,0.37,0.37}{#1}}
\newcommand{\AttributeTok}[1]{\textcolor[rgb]{0.40,0.45,0.13}{#1}}
\newcommand{\BaseNTok}[1]{\textcolor[rgb]{0.68,0.00,0.00}{#1}}
\newcommand{\BuiltInTok}[1]{\textcolor[rgb]{0.00,0.23,0.31}{#1}}
\newcommand{\CharTok}[1]{\textcolor[rgb]{0.13,0.47,0.30}{#1}}
\newcommand{\CommentTok}[1]{\textcolor[rgb]{0.37,0.37,0.37}{#1}}
\newcommand{\CommentVarTok}[1]{\textcolor[rgb]{0.37,0.37,0.37}{\textit{#1}}}
\newcommand{\ConstantTok}[1]{\textcolor[rgb]{0.56,0.35,0.01}{#1}}
\newcommand{\ControlFlowTok}[1]{\textcolor[rgb]{0.00,0.23,0.31}{\textbf{#1}}}
\newcommand{\DataTypeTok}[1]{\textcolor[rgb]{0.68,0.00,0.00}{#1}}
\newcommand{\DecValTok}[1]{\textcolor[rgb]{0.68,0.00,0.00}{#1}}
\newcommand{\DocumentationTok}[1]{\textcolor[rgb]{0.37,0.37,0.37}{\textit{#1}}}
\newcommand{\ErrorTok}[1]{\textcolor[rgb]{0.68,0.00,0.00}{#1}}
\newcommand{\ExtensionTok}[1]{\textcolor[rgb]{0.00,0.23,0.31}{#1}}
\newcommand{\FloatTok}[1]{\textcolor[rgb]{0.68,0.00,0.00}{#1}}
\newcommand{\FunctionTok}[1]{\textcolor[rgb]{0.28,0.35,0.67}{#1}}
\newcommand{\ImportTok}[1]{\textcolor[rgb]{0.00,0.46,0.62}{#1}}
\newcommand{\InformationTok}[1]{\textcolor[rgb]{0.37,0.37,0.37}{#1}}
\newcommand{\KeywordTok}[1]{\textcolor[rgb]{0.00,0.23,0.31}{\textbf{#1}}}
\newcommand{\NormalTok}[1]{\textcolor[rgb]{0.00,0.23,0.31}{#1}}
\newcommand{\OperatorTok}[1]{\textcolor[rgb]{0.37,0.37,0.37}{#1}}
\newcommand{\OtherTok}[1]{\textcolor[rgb]{0.00,0.23,0.31}{#1}}
\newcommand{\PreprocessorTok}[1]{\textcolor[rgb]{0.68,0.00,0.00}{#1}}
\newcommand{\RegionMarkerTok}[1]{\textcolor[rgb]{0.00,0.23,0.31}{#1}}
\newcommand{\SpecialCharTok}[1]{\textcolor[rgb]{0.37,0.37,0.37}{#1}}
\newcommand{\SpecialStringTok}[1]{\textcolor[rgb]{0.13,0.47,0.30}{#1}}
\newcommand{\StringTok}[1]{\textcolor[rgb]{0.13,0.47,0.30}{#1}}
\newcommand{\VariableTok}[1]{\textcolor[rgb]{0.07,0.07,0.07}{#1}}
\newcommand{\VerbatimStringTok}[1]{\textcolor[rgb]{0.13,0.47,0.30}{#1}}
\newcommand{\WarningTok}[1]{\textcolor[rgb]{0.37,0.37,0.37}{\textit{#1}}}

\usepackage{longtable,booktabs,array}
\usepackage{calc} % for calculating minipage widths
% Correct order of tables after \paragraph or \subparagraph
\usepackage{etoolbox}
\makeatletter
\patchcmd\longtable{\par}{\if@noskipsec\mbox{}\fi\par}{}{}
\makeatother
% Allow footnotes in longtable head/foot
\IfFileExists{footnotehyper.sty}{\usepackage{footnotehyper}}{\usepackage{footnote}}
\makesavenoteenv{longtable}
\usepackage{graphicx}
\makeatletter
\newsavebox\pandoc@box
\newcommand*\pandocbounded[1]{% scales image to fit in text height/width
  \sbox\pandoc@box{#1}%
  \Gscale@div\@tempa{\textheight}{\dimexpr\ht\pandoc@box+\dp\pandoc@box\relax}%
  \Gscale@div\@tempb{\linewidth}{\wd\pandoc@box}%
  \ifdim\@tempb\p@<\@tempa\p@\let\@tempa\@tempb\fi% select the smaller of both
  \ifdim\@tempa\p@<\p@\scalebox{\@tempa}{\usebox\pandoc@box}%
  \else\usebox{\pandoc@box}%
  \fi%
}
% Set default figure placement to htbp
\def\fps@figure{htbp}
\makeatother





\setlength{\emergencystretch}{3em} % prevent overfull lines

\providecommand{\tightlist}{%
  \setlength{\itemsep}{0pt}\setlength{\parskip}{0pt}}



 


\KOMAoption{captions}{tableheading}
\makeatletter
\@ifpackageloaded{caption}{}{\usepackage{caption}}
\AtBeginDocument{%
\ifdefined\contentsname
  \renewcommand*\contentsname{Table of contents}
\else
  \newcommand\contentsname{Table of contents}
\fi
\ifdefined\listfigurename
  \renewcommand*\listfigurename{List of Figures}
\else
  \newcommand\listfigurename{List of Figures}
\fi
\ifdefined\listtablename
  \renewcommand*\listtablename{List of Tables}
\else
  \newcommand\listtablename{List of Tables}
\fi
\ifdefined\figurename
  \renewcommand*\figurename{Figure}
\else
  \newcommand\figurename{Figure}
\fi
\ifdefined\tablename
  \renewcommand*\tablename{Table}
\else
  \newcommand\tablename{Table}
\fi
}
\@ifpackageloaded{float}{}{\usepackage{float}}
\floatstyle{ruled}
\@ifundefined{c@chapter}{\newfloat{codelisting}{h}{lop}}{\newfloat{codelisting}{h}{lop}[chapter]}
\floatname{codelisting}{Listing}
\newcommand*\listoflistings{\listof{codelisting}{List of Listings}}
\makeatother
\makeatletter
\makeatother
\makeatletter
\@ifpackageloaded{caption}{}{\usepackage{caption}}
\@ifpackageloaded{subcaption}{}{\usepackage{subcaption}}
\makeatother
\makeatletter
\definecolor{QuartoInternalColor5}{rgb}{0.00,0.40,0.00}
\definecolor{QuartoInternalColor1}{rgb}{0.70,0.17,0.19}
\definecolor{QuartoInternalColor8}{rgb}{0.70,0.49,0.07}
\definecolor{QuartoInternalColor4}{rgb}{0.00,0.45,0.15}
\definecolor{QuartoInternalColor3}{rgb}{0.15,0.56,0.56}
\definecolor{QuartoInternalColor2}{rgb}{0,0,0}
\definecolor{QuartoInternalColor7}{rgb}{0.20,0.40,0.40}
\definecolor{QuartoInternalColor6}{rgb}{0.74,0.74,0.74}
\makeatother
\usepackage{bookmark}
\IfFileExists{xurl.sty}{\usepackage{xurl}}{} % add URL line breaks if available
\urlstyle{same}
\hypersetup{
  pdftitle={Assignment 03 AI Prompts},
  pdfauthor={Ava Godsy},
  colorlinks=true,
  linkcolor={blue},
  filecolor={Maroon},
  citecolor={Blue},
  urlcolor={Blue},
  pdfcreator={LaTeX via pandoc}}


\title{Assignment 03 AI Prompts}
\author{Ava Godsy}
\date{September 22, 2025}
\begin{document}
\maketitle


\section{Cleaning Education column}\label{cleaning-education-column}

\subsection{Prompt}\label{prompt}

using python remove ``/n'' from values in column EDUCATION\_LEVELS\_NAME
in dataframe ``df''

\subsection{Response}\label{response}

To remove ``\n'' (newline characters) from the values in the column
EDUCATION\_LEVELS\_NAME in a DataFrame df, you can use the
.str.replace() method from pandas.

Here's the code:

\begin{Shaded}
\begin{Highlighting}[]
\NormalTok{df[}\StringTok{\textquotesingle{}EDUCATION\_LEVELS\_NAME\textquotesingle{}}\NormalTok{] }\OperatorTok{=}\NormalTok{ df[}\StringTok{\textquotesingle{}EDUCATION\_LEVELS\_NAME\textquotesingle{}}\NormalTok{].}\BuiltInTok{str}\NormalTok{.replace(}\StringTok{\textquotesingle{}}\CharTok{\textbackslash{}n}\StringTok{\textquotesingle{}}\NormalTok{, }\StringTok{\textquotesingle{}\textquotesingle{}}\NormalTok{, regex}\OperatorTok{=}\VariableTok{False}\NormalTok{)}
\end{Highlighting}
\end{Shaded}

Explanation:

str.replace(`\n', '\,`, regex=False) replaces all newline characters
(\n) with an empty string ('\,'), effectively removing them.

regex=False ensures that \n is treated as a literal string (which is
fine in this case since \n is a special character anyway).

If you also want to remove any leading or trailing spaces (common after
removing \n), you can chain .str.strip():

\begin{Shaded}
\begin{Highlighting}[]
\NormalTok{df[}\StringTok{\textquotesingle{}EDUCATION\_LEVELS\_NAME\textquotesingle{}}\NormalTok{] }\OperatorTok{=}\NormalTok{ df[}\StringTok{\textquotesingle{}EDUCATION\_LEVELS\_NAME\textquotesingle{}}\NormalTok{].}\BuiltInTok{str}\NormalTok{.replace(}\StringTok{\textquotesingle{}}\CharTok{\textbackslash{}n}\StringTok{\textquotesingle{}}\NormalTok{, }\StringTok{\textquotesingle{}\textquotesingle{}}\NormalTok{, regex}\OperatorTok{=}\VariableTok{False}\NormalTok{).}\BuiltInTok{str}\NormalTok{.strip()}
\end{Highlighting}
\end{Shaded}

\subsection{Additional Prompt}\label{additional-prompt}

i got an error TypeError: `Column' object is not callable

\subsection{Additional Response}\label{additional-response}

Thanks --- the error:

\begin{Shaded}
\begin{Highlighting}[]
\PreprocessorTok{TypeError}\NormalTok{: }\StringTok{\textquotesingle{}Column\textquotesingle{}} \BuiltInTok{object} \KeywordTok{is} \KeywordTok{not} \BuiltInTok{callable}
\end{Highlighting}
\end{Shaded}

means you're likely using PySpark rather than pandas. In PySpark,
DataFrame columns are represented by Column objects, and they behave
differently than pandas Series.

✅ Fix for PySpark

If you're using PySpark and want to remove \n from a column (like
EDUCATION\_LEVELS\_NAME), use withColumn() and regexp\_replace() from
pyspark.sql.functions.

Here's the correct code:

\begin{Shaded}
\begin{Highlighting}[]
\ImportTok{from}\NormalTok{ pyspark.sql.functions }\ImportTok{import}\NormalTok{ regexp\_replace}

\NormalTok{df }\OperatorTok{=}\NormalTok{ df.withColumn(}\StringTok{"EDUCATION\_LEVELS\_NAME"}\NormalTok{, regexp\_replace(}\StringTok{"EDUCATION\_LEVELS\_NAME"}\NormalTok{, }\StringTok{"}\CharTok{\textbackslash{}n}\StringTok{"}\NormalTok{, }\StringTok{""}\NormalTok{))}
\end{Highlighting}
\end{Shaded}

\section{Salary Distribution by Industry and Employment
Type}\label{salary-distribution-by-industry-and-employment-type}

\subsection{Prompt}\label{prompt-1}

Using Plotly, Create a box plot where: X-axis = NAICS2\_NAME Y-axis =
SALARY\_FROM Group by EMPLOYMENT\_TYPE\_NAME. dataframe = data\_selected

\subsection{Response}\label{response-1}

Here's how you can create a box plot using Plotly in Python, with:

X-axis = NAICS2\_NAME

Y-axis = SALARY\_FROM

Grouped by = EMPLOYMENT\_TYPE\_NAME

Dataframe = data\_selected

✅ Code using Plotly Express:

\begin{Shaded}
\begin{Highlighting}[]
\ImportTok{import}\NormalTok{ plotly.express }\ImportTok{as}\NormalTok{ px}

\NormalTok{fig }\OperatorTok{=}\NormalTok{ px.box(}
\NormalTok{    data\_selected,}
\NormalTok{    x}\OperatorTok{=}\StringTok{"NAICS2\_NAME"}\NormalTok{,}
\NormalTok{    y}\OperatorTok{=}\StringTok{"SALARY\_FROM"}\NormalTok{,}
\NormalTok{    color}\OperatorTok{=}\StringTok{"EMPLOYMENT\_TYPE\_NAME"}\NormalTok{,}
\NormalTok{    title}\OperatorTok{=}\StringTok{"Salary Distribution by Industry and Employment Type"}\NormalTok{,}
\NormalTok{    points}\OperatorTok{=}\StringTok{"all"}\NormalTok{,  }\CommentTok{\# Optional: shows all data points}
\NormalTok{    notched}\OperatorTok{=}\VariableTok{True}   \CommentTok{\# Optional: adds notches to box for median CI}
\NormalTok{)}

\NormalTok{fig.update\_layout(}
\NormalTok{    xaxis\_title}\OperatorTok{=}\StringTok{"Industry (NAICS2)"}\NormalTok{,}
\NormalTok{    yaxis\_title}\OperatorTok{=}\StringTok{"Starting Salary"}\NormalTok{,}
\NormalTok{    boxmode}\OperatorTok{=}\StringTok{"group"}\NormalTok{,  }\CommentTok{\# Groups boxes side by side}
\NormalTok{    xaxis\_tickangle}\OperatorTok{=}\DecValTok{45}  \CommentTok{\# Rotate x{-}axis labels for better readability}
\NormalTok{)}

\NormalTok{fig.show()}
\end{Highlighting}
\end{Shaded}

\subsection{Additional Prompt}\label{additional-prompt-1}

make the height taller, change the colors to purple, blue and green.
make the font garamond

\subsection{Additional Response}\label{additional-response-1}

You can customize the height and colors of your Plotly box plot using
the height parameter and the color\_discrete\_sequence argument.

Here's the updated code with:

Increased height

Custom colors: purple, blue, green

To set the font to Garamond in your Plotly chart, you can specify it in
the update\_layout() method using the font parameter.

✅ Final Updated Code

\begin{Shaded}
\begin{Highlighting}[]
\ImportTok{import}\NormalTok{ plotly.express }\ImportTok{as}\NormalTok{ px}

\NormalTok{fig }\OperatorTok{=}\NormalTok{ px.box(}
\NormalTok{    df\_selected,}
\NormalTok{    x}\OperatorTok{=}\StringTok{"NAICS2\_NAME"}\NormalTok{,}
\NormalTok{    y}\OperatorTok{=}\StringTok{"SALARY\_FROM"}\NormalTok{,}
\NormalTok{    color}\OperatorTok{=}\StringTok{"EMPLOYMENT\_TYPE\_NAME"}\NormalTok{,}
\NormalTok{    title}\OperatorTok{=}\StringTok{"Salary Distribution by Industry and Employment Type"}\NormalTok{,}
\NormalTok{    points}\OperatorTok{=}\StringTok{"all"}\NormalTok{,  }\CommentTok{\# Show all points}
\NormalTok{    notched}\OperatorTok{=}\VariableTok{True}\NormalTok{,  }\CommentTok{\# Notched boxes}
\NormalTok{    height}\OperatorTok{=}\DecValTok{700}\NormalTok{,  }\CommentTok{\# Taller figure}
\NormalTok{    color\_discrete\_sequence}\OperatorTok{=}\NormalTok{[}\StringTok{"purple"}\NormalTok{, }\StringTok{"blue"}\NormalTok{, }\StringTok{"green"}\NormalTok{]  }\CommentTok{\# Custom colors}
\NormalTok{)}

\NormalTok{fig.update\_layout(}
\NormalTok{    xaxis\_title}\OperatorTok{=}\StringTok{"Industry (NAICS2)"}\NormalTok{,}
\NormalTok{    yaxis\_title}\OperatorTok{=}\StringTok{"Starting Salary"}\NormalTok{,}
\NormalTok{    boxmode}\OperatorTok{=}\StringTok{"group"}\NormalTok{,  }\CommentTok{\# Grouped box plots}
\NormalTok{    xaxis\_tickangle}\OperatorTok{=}\DecValTok{45}\NormalTok{,  }\CommentTok{\# Rotate x{-}axis labels}
\NormalTok{    font}\OperatorTok{=}\BuiltInTok{dict}\NormalTok{(}
\NormalTok{        family}\OperatorTok{=}\StringTok{"Garamond, serif"}\NormalTok{,  }\CommentTok{\# Set font to Garamond}
\NormalTok{        size}\OperatorTok{=}\DecValTok{14}  \CommentTok{\# Optional: adjust font size}
\NormalTok{    )}
\NormalTok{)}

\NormalTok{fig.show()}
\end{Highlighting}
\end{Shaded}

\section{Salary Analysis by ONET Occupation Type
Table}\label{salary-analysis-by-onet-occupation-type-table}

\subsection{Prompt}\label{prompt-2}

Using Python, create a table of median ``SALARY'' by ONET\_NAME. also
include a count of records for each ONET\_NAME, call the column
Job\_Postings

\subsection{Response}\label{response-2}

To create a table of the median ``SALARY'' grouped by ``ONET\_NAME'',
and include a count of records (as ``Job\_Postings''), you can use
pandas in Python. Here's how you can do it:

✅ Sample Code

\begin{Shaded}
\begin{Highlighting}[]
\ImportTok{import}\NormalTok{ pandas }\ImportTok{as}\NormalTok{ pd}

\CommentTok{\# Sample: Load your DataFrame}
\CommentTok{\# df = pd.read\_csv(\textquotesingle{}your\_data.csv\textquotesingle{})  \# Or however your DataFrame is created}

\CommentTok{\# Group by ONET\_NAME, calculate median salary and count of job postings}
\NormalTok{result }\OperatorTok{=}\NormalTok{ df.groupby(}\StringTok{\textquotesingle{}ONET\_NAME\textquotesingle{}}\NormalTok{).agg(}
\NormalTok{    SALARY\_Median}\OperatorTok{=}\NormalTok{(}\StringTok{\textquotesingle{}SALARY\textquotesingle{}}\NormalTok{, }\StringTok{\textquotesingle{}median\textquotesingle{}}\NormalTok{),}
\NormalTok{    Job\_Postings}\OperatorTok{=}\NormalTok{(}\StringTok{\textquotesingle{}SALARY\textquotesingle{}}\NormalTok{, }\StringTok{\textquotesingle{}count\textquotesingle{}}\NormalTok{)}
\NormalTok{).reset\_index()}

\CommentTok{\# Optional: Sort by Job\_Postings or SALARY\_Median if needed}
\CommentTok{\# result = result.sort\_values(by=\textquotesingle{}Job\_Postings\textquotesingle{}, ascending=False)}

\CommentTok{\# Display the result}
\BuiltInTok{print}\NormalTok{(result)}

\end{Highlighting}
\end{Shaded}

\begin{Highlighting}
\textcolor{black}{NameError: name 'df' is not defined}
\textcolor{black}{}\textcolor{QuartoInternalColor1}{---------------------------------------------------------------------------}\textcolor{QuartoInternalColor2}{}
\textcolor{QuartoInternalColor2}{}\textcolor{QuartoInternalColor1}{NameError}\textcolor{QuartoInternalColor2}{                                 Traceback (most recent call last)}
\textcolor{QuartoInternalColor2}{}\textcolor{QuartoInternalColor3}{Cell}\textcolor{QuartoInternalColor2}{}\textcolor{QuartoInternalColor3}{ }\textcolor{QuartoInternalColor2}{}\textcolor{QuartoInternalColor4}{In[2]}\textcolor{QuartoInternalColor2}{}\textcolor{QuartoInternalColor4}{, line 10}\textcolor{QuartoInternalColor2}{}
\textcolor{QuartoInternalColor2}{}\textcolor{QuartoInternalColor4}{      4}\textcolor{QuartoInternalColor2}{ }\textcolor{QuartoInternalColor5}{import}\textcolor{QuartoInternalColor2}{}\textcolor{QuartoInternalColor6}{ }\textcolor{QuartoInternalColor2}{}\textcolor{QuartoInternalColor2}{pandas}\textcolor{QuartoInternalColor2}{}\textcolor{QuartoInternalColor6}{ }\textcolor{QuartoInternalColor2}{}\textcolor{QuartoInternalColor5}{as}\textcolor{QuartoInternalColor2}{}\textcolor{QuartoInternalColor6}{ }\textcolor{QuartoInternalColor2}{}\textcolor{QuartoInternalColor2}{pd}\textcolor{QuartoInternalColor2}{}
\textcolor{QuartoInternalColor2}{}\textcolor{QuartoInternalColor4}{      6}\textcolor{QuartoInternalColor2}{ }\textcolor{QuartoInternalColor7}{# Sample: Load your DataFrame}\textcolor{QuartoInternalColor2}{}
\textcolor{QuartoInternalColor2}{}\textcolor{QuartoInternalColor4}{      7}\textcolor{QuartoInternalColor2}{ }\textcolor{QuartoInternalColor7}{# df = pd.read\_csv('your\_data.csv')  # Or however your DataFrame is created}\textcolor{QuartoInternalColor2}{}
\textcolor{QuartoInternalColor2}{}\textcolor{QuartoInternalColor4}{      8}\textcolor{QuartoInternalColor2}{ }
\textcolor{QuartoInternalColor2}{}\textcolor{QuartoInternalColor4}{      9}\textcolor{QuartoInternalColor2}{ }\textcolor{QuartoInternalColor7}{# Group by ONET\_NAME, calculate median salary and count of job postings}\textcolor{QuartoInternalColor2}{}
\textcolor{QuartoInternalColor2}{}\textcolor{QuartoInternalColor4}{---> }\textcolor{QuartoInternalColor2}{}\textcolor{QuartoInternalColor4}{10}\textcolor{QuartoInternalColor2}{ result = }\textcolor{QuartoInternalColor2}{df}\textcolor{QuartoInternalColor2}{.groupby(}\textcolor{QuartoInternalColor8}{'}\textcolor{QuartoInternalColor2}{}\textcolor{QuartoInternalColor8}{ONET\_NAME}\textcolor{QuartoInternalColor2}{}\textcolor{QuartoInternalColor8}{'}\textcolor{QuartoInternalColor2}{).agg(}
\textcolor{QuartoInternalColor2}{}\textcolor{QuartoInternalColor4}{     11}\textcolor{QuartoInternalColor2}{     SALARY\_Median=(}\textcolor{QuartoInternalColor8}{'}\textcolor{QuartoInternalColor2}{}\textcolor{QuartoInternalColor8}{SALARY}\textcolor{QuartoInternalColor2}{}\textcolor{QuartoInternalColor8}{'}\textcolor{QuartoInternalColor2}{, }\textcolor{QuartoInternalColor8}{'}\textcolor{QuartoInternalColor2}{}\textcolor{QuartoInternalColor8}{median}\textcolor{QuartoInternalColor2}{}\textcolor{QuartoInternalColor8}{'}\textcolor{QuartoInternalColor2}{),}
\textcolor{QuartoInternalColor2}{}\textcolor{QuartoInternalColor4}{     12}\textcolor{QuartoInternalColor2}{     Job\_Postings=(}\textcolor{QuartoInternalColor8}{'}\textcolor{QuartoInternalColor2}{}\textcolor{QuartoInternalColor8}{SALARY}\textcolor{QuartoInternalColor2}{}\textcolor{QuartoInternalColor8}{'}\textcolor{QuartoInternalColor2}{, }\textcolor{QuartoInternalColor8}{'}\textcolor{QuartoInternalColor2}{}\textcolor{QuartoInternalColor8}{count}\textcolor{QuartoInternalColor2}{}\textcolor{QuartoInternalColor8}{'}\textcolor{QuartoInternalColor2}{)}
\textcolor{QuartoInternalColor2}{}\textcolor{QuartoInternalColor4}{     13}\textcolor{QuartoInternalColor2}{ ).reset\_index()}
\textcolor{QuartoInternalColor2}{}\textcolor{QuartoInternalColor4}{     15}\textcolor{QuartoInternalColor2}{ }\textcolor{QuartoInternalColor7}{# Optional: Sort by Job\_Postings or SALARY\_Median if needed}\textcolor{QuartoInternalColor2}{}
\textcolor{QuartoInternalColor2}{}\textcolor{QuartoInternalColor4}{     16}\textcolor{QuartoInternalColor2}{ }\textcolor{QuartoInternalColor7}{# result = result.sort\_values(by='Job\_Postings', ascending=False)}\textcolor{QuartoInternalColor2}{}
\textcolor{QuartoInternalColor2}{}\textcolor{QuartoInternalColor4}{     17}\textcolor{QuartoInternalColor2}{ }
\textcolor{QuartoInternalColor2}{}\textcolor{QuartoInternalColor4}{     18}\textcolor{QuartoInternalColor2}{ }\textcolor{QuartoInternalColor7}{# Display the result}\textcolor{QuartoInternalColor2}{}
\textcolor{QuartoInternalColor2}{}\textcolor{QuartoInternalColor4}{     19}\textcolor{QuartoInternalColor2}{ }\textcolor{QuartoInternalColor5}{print}\textcolor{QuartoInternalColor2}{(result)}
\textcolor{QuartoInternalColor2}{}\textcolor{QuartoInternalColor1}{NameError}\textcolor{QuartoInternalColor2}{: name 'df' is not defined}
\end{Highlighting}

\subsection{Additional Prompt}\label{additional-prompt-2}

I got the error TypeError: GroupedData.agg() got an unexpected keyword
argument `SALARY\_Median' \#\# Additional Response

Ah, the error message you're getting:

\begin{Shaded}
\begin{Highlighting}[]
\PreprocessorTok{TypeError}\NormalTok{: GroupedData.agg() got an unexpected keyword argument }\StringTok{\textquotesingle{}SALARY\_Median\textquotesingle{}}
\end{Highlighting}
\end{Shaded}

suggests you're not using pandas, but PySpark instead. In PySpark, the
syntax for .agg() is different --- it doesn't accept keyword arguments
with new column names directly.

✅ Correct PySpark Version

Here's how you can do what you need in PySpark:

\begin{Shaded}
\begin{Highlighting}[]
\ImportTok{from}\NormalTok{ pyspark.sql }\ImportTok{import}\NormalTok{ functions }\ImportTok{as}\NormalTok{ F}

\CommentTok{\# Group by ONET\_NAME and calculate median salary and count of postings}

\CommentTok{\# First, compute Job\_Postings (count) and SALARY\_Median (approximate median using percentile\_approx)}
\NormalTok{result }\OperatorTok{=}\NormalTok{ df.groupBy(}\StringTok{"ONET\_NAME"}\NormalTok{).agg(}
\NormalTok{    F.expr(}\StringTok{"percentile\_approx(SALARY, 0.5)"}\NormalTok{).alias(}\StringTok{"SALARY\_Median"}\NormalTok{),}
\NormalTok{    F.count(}\StringTok{"*"}\NormalTok{).alias(}\StringTok{"Job\_Postings"}\NormalTok{)}
\NormalTok{)}

\CommentTok{\# Show the result}
\NormalTok{result.show()}
\end{Highlighting}
\end{Shaded}

\section{Salary Analysis by ONET Occupation Type
Plot}\label{salary-analysis-by-onet-occupation-type-plot}

\subsection{Prompt}\label{prompt-3}

Using Plotly Create a bubble chart where dataframe = onet\_salary X-axis
= ONET\_NAME Y-axis = Median Salary Size = Job\_Postings Apply custom
colors and font styles.

\subsection{Response}\label{response-3}

ChatGPT said:

Here's how you can create a bubble chart using Plotly with the following
settings:

DataFrame: onet\_salary

X-axis: ONET\_NAME

Y-axis: Median Salary

Bubble Size: Job\_Postings

Custom Colors \& Font Styles: Applied using Plotly's update\_layout

✅ Prerequisites

Ensure you have Plotly installed:

\begin{Shaded}
\begin{Highlighting}[]
\NormalTok{pip install plotly}
\end{Highlighting}
\end{Shaded}

✅ Sample Code

\begin{Shaded}
\begin{Highlighting}[]
\ImportTok{import}\NormalTok{ plotly.express }\ImportTok{as}\NormalTok{ px}

\CommentTok{\# Sample usage (assumes \textquotesingle{}onet\_salary\textquotesingle{} is a DataFrame with relevant columns)}
\NormalTok{fig }\OperatorTok{=}\NormalTok{ px.scatter(}
\NormalTok{    onet\_salary,}
\NormalTok{    x}\OperatorTok{=}\StringTok{"ONET\_NAME"}\NormalTok{,}
\NormalTok{    y}\OperatorTok{=}\StringTok{"Median Salary"}\NormalTok{,}
\NormalTok{    size}\OperatorTok{=}\StringTok{"Job\_Postings"}\NormalTok{,}
\NormalTok{    color}\OperatorTok{=}\StringTok{"Median Salary"}\NormalTok{,  }\CommentTok{\# You can change this to another column if desired}
\NormalTok{    hover\_name}\OperatorTok{=}\StringTok{"ONET\_NAME"}\NormalTok{,}
\NormalTok{    size\_max}\OperatorTok{=}\DecValTok{60}\NormalTok{,}
\NormalTok{    title}\OperatorTok{=}\StringTok{"Bubble Chart of ONET Median Salaries vs Job Postings"}\NormalTok{,}
\NormalTok{)}

\CommentTok{\# Customize layout: fonts, colors, background}
\NormalTok{fig.update\_layout(}
\NormalTok{    title\_font}\OperatorTok{=}\BuiltInTok{dict}\NormalTok{(family}\OperatorTok{=}\StringTok{"Arial Black"}\NormalTok{, size}\OperatorTok{=}\DecValTok{24}\NormalTok{, color}\OperatorTok{=}\StringTok{"darkblue"}\NormalTok{),}
\NormalTok{    font}\OperatorTok{=}\BuiltInTok{dict}\NormalTok{(family}\OperatorTok{=}\StringTok{"Verdana"}\NormalTok{, size}\OperatorTok{=}\DecValTok{14}\NormalTok{, color}\OperatorTok{=}\StringTok{"black"}\NormalTok{),}
\NormalTok{    plot\_bgcolor}\OperatorTok{=}\StringTok{"white"}\NormalTok{,}
\NormalTok{    paper\_bgcolor}\OperatorTok{=}\StringTok{"\#f7f7f7"}\NormalTok{,}
\NormalTok{    xaxis}\OperatorTok{=}\BuiltInTok{dict}\NormalTok{(title}\OperatorTok{=}\StringTok{"ONET Job Titles"}\NormalTok{, tickangle}\OperatorTok{=}\DecValTok{45}\NormalTok{),}
\NormalTok{    yaxis}\OperatorTok{=}\BuiltInTok{dict}\NormalTok{(title}\OperatorTok{=}\StringTok{"Median Salary ($)"}\NormalTok{, gridcolor}\OperatorTok{=}\StringTok{"\#e5e5e5"}\NormalTok{),}
\NormalTok{)}

\CommentTok{\# Show the figure}
\NormalTok{fig.show()}
\end{Highlighting}
\end{Shaded}

\section{Salary by Education Level
Table}\label{salary-by-education-level-table}

\subsection{Prompt}\label{prompt-4}

Create a table using pyspark from dataframe df\_selected where there is
an EDU\_GROUP column, which is a grouping of column
``EDUCATION\_LEVELS\_NAME''. If EDUCATION\_LEVELS\_NAME contains No
Education Listed, GED, or Associate degree then ``Associate's or
lower''. If EDUCATION\_LEVELS\_NAME contains Bachelor's degree then
Bachelor's. If EDUCATION\_LEVELS\_NAME contains Master's degree then
Master's. If EDUCATION\_LEVELS\_NAME contains Ph.D.~or professional
degree then PhD. ALSO Include LOT\_V6\_SPECIALIZED\_OCCUPATION\_NAME
Average\_Salary and MAX\_YEARS\_EXPERIENCE in the table

\subsection{Response}\label{response-4}

To create a table in PySpark from df\_selected with the specified logic
for EDU\_GROUP, and include the requested columns, follow these steps:

✅ Steps Overview

Define EDU\_GROUP using when and col from pyspark.sql.functions

Select required columns

Create a temporary view or save as a table if needed

\begin{Shaded}
\begin{Highlighting}[]
\ImportTok{from}\NormalTok{ pyspark.sql.functions }\ImportTok{import}\NormalTok{ col, when}

\CommentTok{\# Create the EDU\_GROUP column based on EDUCATION\_LEVELS\_NAME}
\NormalTok{df\_with\_edu\_group }\OperatorTok{=}\NormalTok{ df\_selected.withColumn(}
    \StringTok{"EDU\_GROUP"}\NormalTok{,}
\NormalTok{    when(}
\NormalTok{        col(}\StringTok{"EDUCATION\_LEVELS\_NAME"}\NormalTok{).rlike(}\StringTok{"(?i)No Education Listed|GED|Associate"}\NormalTok{), }
        \StringTok{"Associate\textquotesingle{}s or lower"}
\NormalTok{    ).when(}
\NormalTok{        col(}\StringTok{"EDUCATION\_LEVELS\_NAME"}\NormalTok{).rlike(}\StringTok{"(?i)Bachelor"}\NormalTok{),}
        \StringTok{"Bachelor\textquotesingle{}s"}
\NormalTok{    ).when(}
\NormalTok{        col(}\StringTok{"EDUCATION\_LEVELS\_NAME"}\NormalTok{).rlike(}\StringTok{"(?i)Master"}\NormalTok{),}
        \StringTok{"Master\textquotesingle{}s"}
\NormalTok{    ).when(}
\NormalTok{        col(}\StringTok{"EDUCATION\_LEVELS\_NAME"}\NormalTok{).rlike(}\StringTok{"(?i)Ph}\CharTok{\textbackslash{}\textbackslash{}}\StringTok{.D}\CharTok{\textbackslash{}\textbackslash{}}\StringTok{.|professional degree"}\NormalTok{),}
        \StringTok{"PhD"}
\NormalTok{    ).otherwise(}\StringTok{"Other"}\NormalTok{)  }\CommentTok{\# Optional: handle unmatched entries}
\NormalTok{)}

\CommentTok{\# Select required columns}
\NormalTok{final\_df }\OperatorTok{=}\NormalTok{ df\_with\_edu\_group.select(}
    \StringTok{"EDU\_GROUP"}\NormalTok{,}
    \StringTok{"LOT\_V6\_SPECIALIZED\_OCCUPATION\_NAME"}\NormalTok{,}
    \StringTok{"Average\_Salary"}\NormalTok{,}
    \StringTok{"MAX\_YEARS\_EXPERIENCE"}
\NormalTok{)}

\CommentTok{\# Optional: Create a temporary view or save as a table}
\NormalTok{final\_df.createOrReplaceTempView(}\StringTok{"education\_summary\_view"}\NormalTok{)}

\CommentTok{\# Or save as permanent Hive table (optional and requires Hive support)}
\CommentTok{\# final\_df.write.mode("overwrite").saveAsTable("education\_summary")}


\end{Highlighting}
\end{Shaded}

\section{Salary by Education Level
Plot}\label{salary-by-education-level-plot}

\subsection{Prompt}\label{prompt-5}

Plot scatter plot using plotly where colors are EDU\_GROUP (Associate's
or lower is yellow, Bachelor's is green, Master's is blue, and PhD is
purple), dataframe = final\_df, x-axis is MAX\_YEARS\_EXPERIENCE (with
jitter), y-axis is Average\_Salary. Title is Garamond size 24 and all
other text is Garamond size 12. All text is black.

\subsection{Response}\label{response-5}

Here's how you can create a Plotly scatter plot from a PySpark DataFrame
(final\_df) with the following requirements:

✅ Requirements Recap:

x-axis: MAX\_YEARS\_EXPERIENCE with jitter

y-axis: Average\_Salary

Color by: EDU\_GROUP

``Associate's or lower'' → yellow

``Bachelor's'' → green

``Master's'' → blue

``PhD'' → purple

Title: font Garamond, size 24, black

All other text: Garamond, size 12, black

✅ Full Code (with Explanation)

\begin{Shaded}
\begin{Highlighting}[]
\ImportTok{import}\NormalTok{ plotly.express }\ImportTok{as}\NormalTok{ px}
\ImportTok{import}\NormalTok{ numpy }\ImportTok{as}\NormalTok{ np}

\CommentTok{\# Step 1: Convert PySpark DataFrame to Pandas}
\NormalTok{pdf }\OperatorTok{=}\NormalTok{ final\_df.toPandas()}

\CommentTok{\# Step 2: Add jitter to MAX\_YEARS\_EXPERIENCE}
\NormalTok{np.random.seed(}\DecValTok{0}\NormalTok{)}
\NormalTok{jitter\_strength }\OperatorTok{=} \FloatTok{0.3}
\NormalTok{pdf[}\StringTok{\textquotesingle{}JITTERED\_EXPERIENCE\textquotesingle{}}\NormalTok{] }\OperatorTok{=}\NormalTok{ pdf[}\StringTok{\textquotesingle{}MAX\_YEARS\_EXPERIENCE\textquotesingle{}}\NormalTok{] }\OperatorTok{+}\NormalTok{ np.random.uniform(}
    \OperatorTok{{-}}\NormalTok{jitter\_strength, jitter\_strength, size}\OperatorTok{=}\BuiltInTok{len}\NormalTok{(pdf)}
\NormalTok{)}

\CommentTok{\# Step 3: Define custom color mapping}
\NormalTok{color\_map }\OperatorTok{=}\NormalTok{ \{}
    \StringTok{"Associate\textquotesingle{}s or lower"}\NormalTok{: }\StringTok{\textquotesingle{}yellow\textquotesingle{}}\NormalTok{,}
    \StringTok{"Bachelor\textquotesingle{}s"}\NormalTok{: }\StringTok{\textquotesingle{}green\textquotesingle{}}\NormalTok{,}
    \StringTok{"Master\textquotesingle{}s"}\NormalTok{: }\StringTok{\textquotesingle{}blue\textquotesingle{}}\NormalTok{,}
    \StringTok{"PhD"}\NormalTok{: }\StringTok{\textquotesingle{}purple\textquotesingle{}}
\NormalTok{\}}

\CommentTok{\# Step 4: Create the Plotly scatter plot}
\NormalTok{fig }\OperatorTok{=}\NormalTok{ px.scatter(}
\NormalTok{    pdf,}
\NormalTok{    x}\OperatorTok{=}\StringTok{\textquotesingle{}JITTERED\_EXPERIENCE\textquotesingle{}}\NormalTok{,}
\NormalTok{    y}\OperatorTok{=}\StringTok{\textquotesingle{}Average\_Salary\textquotesingle{}}\NormalTok{,}
\NormalTok{    color}\OperatorTok{=}\StringTok{\textquotesingle{}EDU\_GROUP\textquotesingle{}}\NormalTok{,}
\NormalTok{    color\_discrete\_map}\OperatorTok{=}\NormalTok{color\_map,}
\NormalTok{    title}\OperatorTok{=}\StringTok{"Average Salary vs. Max Years of Experience by Education Group"}\NormalTok{,}
\NormalTok{    labels}\OperatorTok{=}\NormalTok{\{}
        \StringTok{\textquotesingle{}JITTERED\_EXPERIENCE\textquotesingle{}}\NormalTok{: }\StringTok{\textquotesingle{}Max Years of Experience (jittered)\textquotesingle{}}\NormalTok{,}
        \StringTok{\textquotesingle{}Average\_Salary\textquotesingle{}}\NormalTok{: }\StringTok{\textquotesingle{}Average Salary\textquotesingle{}}\NormalTok{,}
        \StringTok{\textquotesingle{}EDU\_GROUP\textquotesingle{}}\NormalTok{: }\StringTok{\textquotesingle{}Education Group\textquotesingle{}}
\NormalTok{    \},}
\NormalTok{    opacity}\OperatorTok{=}\FloatTok{0.7}
\NormalTok{)}

\CommentTok{\# Step 5: Update layout with Garamond font and sizes}
\NormalTok{fig.update\_layout(}
\NormalTok{    title\_font}\OperatorTok{=}\BuiltInTok{dict}\NormalTok{(family}\OperatorTok{=}\StringTok{\textquotesingle{}Garamond\textquotesingle{}}\NormalTok{, size}\OperatorTok{=}\DecValTok{24}\NormalTok{, color}\OperatorTok{=}\StringTok{\textquotesingle{}black\textquotesingle{}}\NormalTok{),}
\NormalTok{    font}\OperatorTok{=}\BuiltInTok{dict}\NormalTok{(family}\OperatorTok{=}\StringTok{\textquotesingle{}Garamond\textquotesingle{}}\NormalTok{, size}\OperatorTok{=}\DecValTok{12}\NormalTok{, color}\OperatorTok{=}\StringTok{\textquotesingle{}black\textquotesingle{}}\NormalTok{),}
\NormalTok{    legend\_title\_font}\OperatorTok{=}\BuiltInTok{dict}\NormalTok{(family}\OperatorTok{=}\StringTok{\textquotesingle{}Garamond\textquotesingle{}}\NormalTok{, size}\OperatorTok{=}\DecValTok{12}\NormalTok{, color}\OperatorTok{=}\StringTok{\textquotesingle{}black\textquotesingle{}}\NormalTok{),}
\NormalTok{    legend\_font}\OperatorTok{=}\BuiltInTok{dict}\NormalTok{(family}\OperatorTok{=}\StringTok{\textquotesingle{}Garamond\textquotesingle{}}\NormalTok{, size}\OperatorTok{=}\DecValTok{12}\NormalTok{, color}\OperatorTok{=}\StringTok{\textquotesingle{}black\textquotesingle{}}\NormalTok{)}
\NormalTok{)}

\CommentTok{\# Step 6: Show the figure}
\NormalTok{fig.show()}
\end{Highlighting}
\end{Shaded}

\section{Salary by Remote Work Type
Histograms}\label{salary-by-remote-work-type-histograms}

\subsection{Prompt}\label{prompt-6}

create a plotly histogram where the x-axis is ``MAX\_YEARS\_EXPERIENCE''
split into groups of 1, and the y axis is ``Average\_Salary'', dataframe
is remote\_df, and group by REMOTE\_TYPE\_NAME

\subsection{Response}\label{response-6}

To create a Plotly histogram where:

x = MAX\_YEARS\_EXPERIENCE (binned by 1 year),

y = Average\_Salary,

grouped by = REMOTE\_TYPE\_NAME,

you'll want to use Plotly Express with histfunc=`avg' so that it
computes the average salary per bin, and use nbins or manually bin the
experience column.

✅ Option 1: Use Plotly's Built-in Histogram Aggregation

This is the simplest approach:

\begin{Shaded}
\begin{Highlighting}[]
\ImportTok{import}\NormalTok{ plotly.express }\ImportTok{as}\NormalTok{ px}

\NormalTok{fig }\OperatorTok{=}\NormalTok{ px.histogram(}
\NormalTok{    remote\_df,}
\NormalTok{    x}\OperatorTok{=}\StringTok{"MAX\_YEARS\_EXPERIENCE"}\NormalTok{,}
\NormalTok{    y}\OperatorTok{=}\StringTok{"Average\_Salary"}\NormalTok{,}
\NormalTok{    color}\OperatorTok{=}\StringTok{"REMOTE\_TYPE\_NAME"}\NormalTok{,}
\NormalTok{    histfunc}\OperatorTok{=}\StringTok{"avg"}\NormalTok{,               }\CommentTok{\# compute average salary per bin}
\NormalTok{    nbins}\OperatorTok{=}\BuiltInTok{int}\NormalTok{(remote\_df[}\StringTok{\textquotesingle{}MAX\_YEARS\_EXPERIENCE\textquotesingle{}}\NormalTok{].}\BuiltInTok{max}\NormalTok{()) }\OperatorTok{+} \DecValTok{1}\NormalTok{,}
\NormalTok{    title}\OperatorTok{=}\StringTok{"Average Salary by Years of Experience and Remote Type"}\NormalTok{,}
\NormalTok{    labels}\OperatorTok{=}\NormalTok{\{}
        \StringTok{\textquotesingle{}MAX\_YEARS\_EXPERIENCE\textquotesingle{}}\NormalTok{: }\StringTok{\textquotesingle{}Max Years of Experience\textquotesingle{}}\NormalTok{,}
        \StringTok{\textquotesingle{}Average\_Salary\textquotesingle{}}\NormalTok{: }\StringTok{\textquotesingle{}Average Salary\textquotesingle{}}\NormalTok{,}
        \StringTok{\textquotesingle{}REMOTE\_TYPE\_NAME\textquotesingle{}}\NormalTok{: }\StringTok{\textquotesingle{}Remote Work Type\textquotesingle{}}
\NormalTok{    \},}
\NormalTok{    barmode}\OperatorTok{=}\StringTok{\textquotesingle{}group\textquotesingle{}}  \CommentTok{\# group bars side by side}
\NormalTok{)}

\NormalTok{fig.update\_layout(}
\NormalTok{    xaxis\_title}\OperatorTok{=}\StringTok{"Max Years of Experience"}\NormalTok{,}
\NormalTok{    yaxis\_title}\OperatorTok{=}\StringTok{"Average Salary"}\NormalTok{,}
\NormalTok{    bargap}\OperatorTok{=}\FloatTok{0.2}
\NormalTok{)}

\NormalTok{fig.show()}
\end{Highlighting}
\end{Shaded}

\subsection{Additional Prompt}\label{additional-prompt-3}

I got the error TypeError: `Column' object is not callable

\subsection{Additional Response}\label{additional-response-2}

Thanks --- that error suggests you're using PySpark DataFrame
(remote\_df) directly with Plotly, which doesn't work because Plotly
expects a Pandas DataFrame, not a PySpark one.

The error likely came from this line:

\begin{Shaded}
\begin{Highlighting}[]
\NormalTok{nbins}\OperatorTok{=}\BuiltInTok{int}\NormalTok{(remote\_df[}\StringTok{\textquotesingle{}MAX\_YEARS\_EXPERIENCE\textquotesingle{}}\NormalTok{].}\BuiltInTok{max}\NormalTok{()) }\OperatorTok{+} \DecValTok{1}
\end{Highlighting}
\end{Shaded}

In PySpark, remote\_df{[}`MAX\_YEARS\_EXPERIENCE'{]} returns a Column
object, not a value --- which leads to the TypeError: `Column' object is
not callable.

✅ Fix: Convert to Pandas First

You must convert your PySpark DataFrame to Pandas before plotting with
Plotly:

✔️ Step-by-step Fix

\begin{Shaded}
\begin{Highlighting}[]
\CommentTok{\# Step 1: Convert to Pandas}
\NormalTok{pdf }\OperatorTok{=}\NormalTok{ remote\_df.toPandas()}

\CommentTok{\# Step 2: Create plot using Plotly}
\ImportTok{import}\NormalTok{ plotly.express }\ImportTok{as}\NormalTok{ px}

\NormalTok{fig }\OperatorTok{=}\NormalTok{ px.histogram(}
\NormalTok{    pdf,}
\NormalTok{    x}\OperatorTok{=}\StringTok{"MAX\_YEARS\_EXPERIENCE"}\NormalTok{,}
\NormalTok{    y}\OperatorTok{=}\StringTok{"Average\_Salary"}\NormalTok{,}
\NormalTok{    color}\OperatorTok{=}\StringTok{"REMOTE\_TYPE\_NAME"}\NormalTok{,}
\NormalTok{    histfunc}\OperatorTok{=}\StringTok{"avg"}\NormalTok{,}
\NormalTok{    nbins}\OperatorTok{=}\BuiltInTok{int}\NormalTok{(pdf[}\StringTok{\textquotesingle{}MAX\_YEARS\_EXPERIENCE\textquotesingle{}}\NormalTok{].}\BuiltInTok{max}\NormalTok{()) }\OperatorTok{+} \DecValTok{1}\NormalTok{,}
\NormalTok{    barmode}\OperatorTok{=}\StringTok{\textquotesingle{}group\textquotesingle{}}\NormalTok{,}
\NormalTok{    title}\OperatorTok{=}\StringTok{"Average Salary by Years of Experience and Remote Type"}\NormalTok{,}
\NormalTok{    labels}\OperatorTok{=}\NormalTok{\{}
        \StringTok{\textquotesingle{}MAX\_YEARS\_EXPERIENCE\textquotesingle{}}\NormalTok{: }\StringTok{\textquotesingle{}Max Years of Experience\textquotesingle{}}\NormalTok{,}
        \StringTok{\textquotesingle{}Average\_Salary\textquotesingle{}}\NormalTok{: }\StringTok{\textquotesingle{}Average Salary\textquotesingle{}}\NormalTok{,}
        \StringTok{\textquotesingle{}REMOTE\_TYPE\_NAME\textquotesingle{}}\NormalTok{: }\StringTok{\textquotesingle{}Remote Work Type\textquotesingle{}}
\NormalTok{    \}}
\NormalTok{)}

\NormalTok{fig.update\_layout(}
\NormalTok{    xaxis}\OperatorTok{=}\BuiltInTok{dict}\NormalTok{(dtick}\OperatorTok{=}\DecValTok{1}\NormalTok{),}
\NormalTok{    yaxis\_title}\OperatorTok{=}\StringTok{"Average Salary"}\NormalTok{,}
\NormalTok{    bargap}\OperatorTok{=}\FloatTok{0.2}
\NormalTok{)}

\NormalTok{fig.show()}
\end{Highlighting}
\end{Shaded}





\end{document}
